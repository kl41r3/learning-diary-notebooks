%================= test.tex =================
\documentclass[11pt,a4paper]{article}

%---------- 宏包 ----------
\usepackage[UTF8]{ctex}          % 中文
\usepackage{amsmath,amssymb}     % 数学
\usepackage{graphicx}            % 插图
\usepackage{booktabs}            % 三线表
\usepackage{listings}            % 代码
\usepackage{xcolor}              % 高亮
\usepackage{hyperref}            % 超链接

%---------- 代码样式 ----------
\lstset{
  basicstyle=\ttfamily\small,
  keywordstyle=\color{blue!70},
  commentstyle=\color{gray},
  stringstyle=\color{orange},
  numbers=left,
  numberstyle=\tiny,
  breaklines=true,
  frame=single,
  language=Python
}

%---------- 正文 ----------
\begin{document}

\title{$P(X=k) = \frac{\lambda^k e^{-\lambda}}{k!}$}
\author{Kimi}
\date{\today}
\maketitle

\begin{abstract}
这是一份用于测试 LaTeX 各项功能的简短文档,涵盖中文、公式、表格、图片、代码与参考文献。
\end{abstract}

\tableofcontents
\newpage

\section{中文支持}
XeLaTeX 编译可直接输入中文:你好,世界!

\section{数学公式}
行内公式 $E=mc^2$,以及编号公式:
\begin{equation}\label{eq:eu}
\int_{-\infty}^{+\infty} e^{-x^2}\,\mathrm{d}x = \sqrt{\pi}.
\P(N(t)=k) = \frac{(\lambda t)^k e^{-\lambda t}}{k!}
\end{equation}
式~\eqref{eq:eu} 可通过 {\tt \textbackslash eqref} 引用。

\section{表格与图片}
表~\ref{tab:demo} 是一个三线表示例。

\begin{table}[ht]
\centering
\caption{测试表格}
\begin{tabular}{@{}lcr@{}}
\toprule
姓名 & 语文 & 数学 \\ \midrule
张三 & 88 & 95 \\
李四 & 92 & 87 \\ \bottomrule
\end{tabular}
\label{tab:demo}
\end{table}

图~\ref{fig:logo} 是 Overleaf 官方 logo(需联网,若编译失败可注释掉)。

\begin{figure}[ht]
\centering
\caption{Overleaf Logo}
\label{fig:logo}
\end{figure}

\section{代码高亮}
下面给出快速排序的 Python 实现:

\begin{lstlisting}[language=Python]
def quicksort(arr):
    if len(arr) <= 1:
        return arr
    pivot = arr[len(arr)//2]
    left  = [x for x in arr if x < pivot]
    mid   = [x for x in arr if x == pivot]
    right = [x for x in arr if x > pivot]
    return quicksort(left) + mid + quicksort(right)
\end{lstlisting}

\section{参考文献}
本文引用了一本经典教材~\cite{lamport1994latex}。

\bibliographystyle{plain}
\begin{thebibliography}{1}
\bibitem{lamport1994latex}
Leslie Lamport.
\textit{\LaTeX: A Document Preparation System}. Addison-Wesley, 1994.
\end{thebibliography}

\end{document}
%===========================================